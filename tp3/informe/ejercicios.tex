\section{Implementación}

En esta sección comentaremos brevemente tanto la implementación del algoritmo de
elección de líder distribuido como las decisiones tomadas durante
el desarrollo del mismo.

\subsection{Algoritmo}

Se tiene a los procesos organizados en forma de anillo, donde cada proceso
conoce únicamente su número de id (además de saber si es el último del anillo
o no), y se busca elegir a uno de ellos como líder (en este caso, el líder
deberá ser el que tenga el número de id más grande). Además, cada proceso
guarda una variable \verb|status|, cuyo valor indica si la última elección
que circuló lo eligió como líder o no.
El algoritmo distribuido para lograr esto consiste en lo siguiente:

\begin{itemize}

    \item Al comienzo de su ejecución, cada proceso decidirá
    pseudoaleatoriamente si comenzará o no una nueva
    elección de líder; en caso afirmativo, enviará un mensaje de la forma
    \verb|<id, id>| al siguiente proceso en el anillo para iniciar la elección.
    Para todos los mensajes de esta forma, el primer componente de la tupla 
    indica qué proceso inició la elección, y el segundo indica el candidato
    a líder actual de la elección.

    \item Cada proceso, durante su tiempo de vida, esperará por la
    llegada de nuevos mensajes de la forma \verb|<id, cl>|. Al recibir un 
    mensaje, ejecutará los siguientes pasos:

    \begin{itemize}

        \item Setear \verb|status = NO_LIDER|, ya que hay una elección
        en curso.

        \item Si el id recibido es igual al id del proceso, la elección 
        circuló por todo el anillo, eligiendo a \verb|cl| como líder.
        Si \verb|cl| es igual al id del proceso, actualizar
        \verb|status = LIDER|; caso contrario, mandarle al siguiente proceso
        el mensaje \verb|<cl, cl>|, para avisarle al proceso líder que es el
        ganador (que está más adelante en el anillo y todavía no sabe que él
        es el líder)

        \item Si el id recibido no es igual al id del proceso, la elección
        todavía no terminó, y debe enviarse al siguiente proceso el mensaje
        \verb|<id, nuevo_cl>|, siendo \verb|nuevo_cl| el máximo entre \verb|cl|
        y el id del proceso.

    \end{itemize}
    
    \item Una vez agotado el tiempo de ejecución, cada proceso reportará si él
    es el líder o no, en función del valor que tenga \verb|status| al final.

\end{itemize}

Al utilizar las funciones de comunicación no bloqueante, después de enviar cada
mensaje de la forma \verb|<id, cl>|, el proceso emisor del mensaje deberá
esperar a que el proceso receptor le envíe un ACK ya que, según el enunciado,
los procesos pueden irse del anillo sin notificarle a ninguno de los otros
procesos del anillo. Esto implica que, si el proceso emisor no recibe un ACK
pasada una cantidad fija de tiempo desde el envío, deberá asumir que el proceso
receptor ha salido del anillo y volver a intentar con el siguiente proceso del
anillo, operatoria que se repetirá hasta haber obtenido un ACK de algún proceso
del anillo. Esto puede traer una serie de problemas, que detallaremos a
continuación, junto con las estrategias que proponemos para solucionarlos.

\subsection{Decisiones de implementación}

Un potencial problema en el algoritmo descripto es el siguiente: cuando un
proceso $P_{i}$ envía un mensaje \verb|<id, cl>| al siguiente, podría haber
otro proceso $P_{j}$ tratando de enviarle un mensaje a $P_{i}$, ya que puede
haber elecciones de líder concurrentes dentro del anillo. El problema radica en
que, como $P_{i}$ está esperando por el ACK de su propio siguiente, $P{j}$ nunca
recibiría el ACK que está esperando de $P_{i}$, con lo cual $P_{j}$ concluiría
(erróneamente) que $P_{i}$ salió del anillo, incrementaría el índice del proceso
al que le mandará el mensaje, y en el peor caso podría intentar enviarle el
mensaje a un id de proceso inválido (si al incrementar el índice se excede
de la cantidad total de procesos), lo cual hace que todos los procesos del
anillo terminen su ejecución con error (por el envío de un mensaje a un rank
inválido). \\

Para resolver este problema, decidimos que si un proceso recibe mensajes de la
forma \verb|<id, cl>| mientras está esperando por el ACK de otro proceso, los
conteste inmediatamente con el ACK correspondiente (para impedir que crean que el
proceso salió del anillo), pero que ignore el contenido de la tupla
\verb|<id, cl>| recibida; es decir, que los mensajes recibidos mientras se espera
por un ACK no se propaguen por el anillo.
Tomamos esta decisión porque consideramos que, si bien se podrían
perder algunas elecciones de líder de los procesos anteriores a $P_{i}$,
la elección del proceso que está haciendo girar $P_{i}$ sigue girando por el
anillo, con lo cual siempre hay por lo menos una elección que hace el recorrido
completo. \\

Otro problema posible se presenta en el caso borde en el cual se intenta elegir un líder en
un anillo que contiene un solo proceso vivo. En este caso, al iniciar la
elección, se tiene un proceso que está esperando por un ACK que él mismo debe
enviarse. Para resolver este problema, al momento de enviar un inicio de
elección, verificamos si el destinatario del mensaje tiene el mismo id que el
proceso que está iniciando la elección; en caso afirmativo, se cortará
manualmente (mediante la sentencia \verb|break|) el ciclo de espera por el ACK, ya que
consideramos que no tiene sentido que un proceso espere por una confirmación de
que él mismo está vivo.
